% ---------------------------------------------------------------------------

\section{Glossary}
\label{sec:glossary}

\begin{labeling}[~--~]{\hspace{5em}}
\item[CDF] 
  cumulative distribution function.

\item[center]
  ``typical point'' of distribution (``near the mode'').

\item[HR]
  hazard rate (or failure rate).

\item[inverse local concavity]
  local concavity of inverse PDF $f^{-1}(y)$ expressed in term of \\
  $x = f^{-1}(y)$. Is is given by 
  \[
  \mathrm{ilc}_f(x) = 1 + x\,f''(x) / f'(x)\;.
  \]

\item[mode]
  maximum of PDF.

\item[local concavity]
  maximum value of $c$ such that PDF $f(x)$ is $T_c$-concave.
  Is is given by
  \[
  \mathrm{lc}_f(x) = 1 - f''(x)\,f(x) / f'(x)^2\;.
  \]

\item[log-concave]
  a PDF $f(x)$ (and hence the corresponding distribution) is called 
  \emph{log-concave} if $\log(f(x))$ is concave, i.e., if
  $(\log(f(x))''\leq 0$. See also \emph{$T_0$-concave}.

  For discrete distributions, a PMF $p$ is log-concave if and only if
  \[
  p_i \geq\sqrt{p_{i-1}\,p_{i+1}}
  \quad\mbox{for all $i$.}
  \]

\item[PDF]
  probability density function.

\item[PMF]
  probability mass function.

\item[PV]
  (finite) probability vector.

\item[URNG]
  uniform random number generator.

\item[$U(a,b)$]
  continuous uniform distribution on the interval $(a,b)$.

\item[$T_c$-concave]
  a PDF $f(x)$ is called \emph{$T$-concave} if the transformed function
  $T(f(x))$ is concave. 
  We only deal with transformations $T_c$, where
  \begin{center}
    \begin{tabular}{l|l}
      c         & transformation \\
      \hline
      $c=0$     & $T_0(x) = \log(x)$ \\
      $c=-1/2$  & $T_{-1/2}(x) = -1/\sqrt{x}$ \\
      $c\not=0$ & $T_c(x) = \mathrm{sgn}(c) \cdot x^c$ \\
    \end{tabular}
  \end{center}
  In particular, a PDF $f(x)$ is $T_c$-concave when its local
  concavity is less than $c$, i.e.,
  $\mathrm{lc}_f(x) \leq c$.

\item[$u$-error]
  for a given approximate inverse CDF $X=G^{-1}(U)$ the $u$-error is
  given as
  \[ \mbox{$u$-error} = |U-F(G^{-1}(U))| \]
  where $F$ denotes the exact CDF.
  Goodness-of-fit tests like the Kolmogorov-Smirnov test or the
  chi-squared test look at this type of error.

\item[$u$-resolution]
  the maximal tolerated $u$-error for an approximate inverse CDF.

\item[$x$-error]
  for a given approximate inverse CDF $X=G^{-1}(U)$ the $x$-error is
  given as 
  \[ \mbox{$x$-error} = |F^{-1}(U)-G^{-1}(U)| \]
  where $F^{-1}$ denotes the exact inverse CDF.
  The $x$-error measure the deviation of $G^{-1}(U)$
  from the exact result.
  Notice that we have to distinguish between \emph{absolute} and
  \emph{relative} $x$-error. In UNU.RAN we use the absolute $x$-error
  near $0$ and the relative $x$-error otherwise,
  see \autoref{sec:inverror} for more details.

\item[$x$-resolution]
  the maximal tolerated $x$-error for an approximate inverse CDF.

\end{labeling}

% ---------------------------------------------------------------------------
